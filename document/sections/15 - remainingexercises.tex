\section{Esercizi rimanenti}
\subsection{Calcolo dimensioni dello schema logico a partire dallo schema di fatto}
L'esercizio fornisce uno schema di fatto, lo scopo e fare la progettazione logica creando la FT e le DT. In aggiunta è necessario calcolare quanti byte occupa la soluzione proposta. In alcuni esercizi viene chiesto di trovare la soluzione che occupa meno spazio, in altre di trovare più di una soluzione o comunque le principali. Per trovare la soluzione meno costosa spesso basta fare snowflaking il più possibile.\newline

\noindent Una volta identificate le DT il calcolo della dimensione è la sommatoria del peso di tutte le tabelle. Per calcolare il peso di una tabella la formula da utilizzare è $peso_tupla \cdot cardinalità$.\newline

\noindent La parte difficile di questo esercizio è capire come strutturare le DT.

\subsubsection{Gerarchia ricorsiva}
Anche secondo la teoria le soluzioni sono due:
\begin{itemize}
	\item inserire una foreign key alla stessa DT in più;
	\item creare una tabella con legami di parentela.
\end{itemize}
La tabella in aggiunta ha i due id del nodo padre e figlio come chiave. Non vanno inseriti solo i legami tra padre e figlio, va inserito ogni legame con ogni nodo possibile, quindi il nodo con se stesso, nodo con figli, nodo con nipoti, nodo con bisnipoti etc. Non vanno inserite entry se i legami non esistono. La tuple è formata da ID padre, ID figlio, livello e se è foglia (booleano).
\subsection{Calcolo spazio dimensioni degeneri}
L'esercizio chiede di calcolare quanto spazio occupano determinate soluzioni di uno schema logico in cui vengono prese in considerazione solo dimensioni degeneri.
\noindent Ricordiamo dalla teoria che ci sono tre modi per trattare una dimensione degenere:
\begin{itemize}
	\item Inserirla in Fact Table;
	\item Creare una DT apposita;
	\item Inserirla in una Junk Table insieme ad altre dimensioni degeneri.
\end{itemize}
Ovviamente trovare la soluzione che minimizza lo spazio utilizzato diventa un problema combinatorio perché ogni dimensione degenere può essere trattata diversamente. Il consiglio è di fare come segue:
\begin{itemize}
	\item Per il momento ignoriamo la Junk Table come possibile soluzione;
	\item Per ogni dimensione degenere facciamo la seguente valutazione, trattandola come se fosse la sola e unica dimensione degenere:
	\begin{itemize}
		\item Creiamo la DT, aggiungendo il surrogato sia in FT che in DT, solitamente negli esercizi ha un peso di 4byte ma comunque c'è scritto. Calcoliamo per entrambe le tabelle il peso utilizzando la formula $peso_tupla \cdot cardinalità$.
		\item Facciamo lo stesso calcolo utilizzando la stessa formula ma stavolta mettendo il dato direttamente in FT.
	\end{itemize}
	\item Il numero più basso indica la soluzione più adatta alla dimensione degenere presa in considerazione. Si prosegue così per tutte le rimanenti.
	\item Si calcola la dimensione totale della soluzione migliore trovata utilizzando anche i valori che abbiamo già trovato per le DT, si calcola quella della FT e si sommano. Abbiamo il peso della soluzione ottima se non consideriamo la Junk Table.
	\item Ora però la Junk Table la dobbiamo considerare. La cardinalità della Junk equivale a tutte le possibili combinazioni delle dimensioni degeneri, quindi facciamo il prodotto di tutte le cardinalità. Abbiamo il surrogato sulla DT e sulla FT.
	\item È sconsigliato provare a creare soluzioni ibride in cui alcune dimensioni stanno in junk ed altre no. L'unico caso può essere se una dimensione ha una cardinalità veramente fuori scala rispetto alle altre e quindi può convenire isolarla, ma è veramente un caso limite.
	\item Confrontando le due soluzioni avremo quasi sicuramente un vincitore.
\end{itemize}
\subsection{Progettazione logica spinta}
Si tratta di un esercizio in cui si chiede di effettuare la progettazione logica, ma ci sono costrutti di varia natura e si chiede di calcolare l'occupazione di spazio. Qua bisogna esser bravi a trattare in modo corretto ogni costrutto. Non si chiede di minimizzare nulla, basta fornire una soluzione adeguata. Il calcolo dello spazio si effettua nello stesso modo degli esercizi precedenti. Una rapida carrellata di come si possono trattare i vari costrutti:
\begin{itemize}
	\item \textbf{Gerarchie condivise}: creare sempre uno snowflake;
	\item \textbf{Archi opzionali}: nessun impatto;
	\item \textbf{Gerarchia incompleta}: nessun impatto;
	\item \textbf{Arco multiplo}: creare una bridge table. Ci sono diverse scelte, possibili, utilizzando solo foreign key o facendo snowflake. Mai scordarsi di aggiungere il peso. La cardinalità della Bridge Table nel caso peggiore è il prodotto delle due cardinalità, ma spesso nella consegna c'è scritto un modo per stimarla meglio.
\end{itemize}
\subsection{Reticoli multidimensionali}
Dato uno schema di fatto viene chiesto di disegnare il reticolo multidimensionale. Si parte dal group by più fine , quindi tutti gli attributi che si trovano uscendo dal fatto. Se da un attributo parte più di un attributo, ad esempio da A partono B e C, il group by che si ottiene da A è BC e non solo B o solo C. Si continua ad aggregare finché non si arriva al group by vuoto.
\subsubsection{Gerarchie condivise}
Le gerarchie condivise vanno trattate come se fossero separate. Quindi diamo un ruolo ai collegamenti della gerarchia e poi li trattiamo come se fossero separate. Il ruolo lo utilizziamo per identificare la dimensione.
\subsubsection{Convergenza}
A differenza della condivisione non ci sono due istanze dell'attributo ma uno solo, quindi tutti gli attributi convergono semplicemente allo stesso attributo.
\subsubsection{Attributo Cross-dimensionale}
Le due dimensioni che creano l'attributo cross-dimensionale collassano entrambe sull'attributo cross. Quindi  se \textbf{a} e \textbf{b} portano a  \textbf{c}, il group by {\textbf{ab}} collassa direttamente in {\textbf{c}}.
\subsubsection{Arco Multiplo}
Essendo l'associazione tra di due punti dell'arco multiplo una molti a molti, è possibile (anche se ha poco senso) avere un group by che comprende entrambi i valori. Quindi un arco multiplo da cui parte \textbf{a} che va verso \textbf{b} passa da \textbf{a} ad \textbf{ab}  che poi si divide in \textbf{a} e \textbf{b} che convergono a {}. In pratica sviluppare l'arco multiplo equivale a fare un push-down dell'arco, facendolo diventare un normale arco che parte dal livello inferiore rispetto all'origine.
\subsection{Miglior punto per Snowflake}
Lo scopo è trovare euristicamente il miglior punto per fare snowflake su uno schema di fatto. Il punto migliore è (tranne in casi particolari) sempre quello in cui il rapporto tra le cardinali è maggiore. Una volta trovato il punto migliore è sufficiente effettuare la classica progettazione logica. Se l’esercizio chiede il calcolo del peso della soluzione trovata è sufficiente utilizzare la solita formula $peso_tupla \cdot cardinalità$.
\subsection{Trovare numero eventi primari entro cui junk table è migliore}
Il procedimento è tale identico all’esercizio in cui bisogna trovare la soluzione più conveniente, ma la carnalità va rimpiazzata con una X che è il valore da trovare. Tra le due soluzioni migliori trovate basterà fare una diseguaglianza e trovare il valore di X.
\subsection{Viste candidate}
In questi esercizi non è richiesto di rappresentare tutto il reticolo, ma può essere utile disegnarne una parte.  Lo scopo è trovare le viste che ha senso materializzare dati alcuni group by. Le viste che comprendono i group by sono sicuramente candidate. La prima cosa da fare è scrivere i group by è cercare di “metterli in fila” ossia capire se esistono legami di parentela. Non sempre questi legami esistono, spesso non è possibile arrivare da una vista all’altro con dei passi di aggregazione. Lo scopo è cercare di trovare una specie di “minimo comune multiplo” tra i group by che non sono collegati tra loro (ove possibile) in modo da riuscire a costruire alberi di dipendenze. Per ogni coppia in pratica bisogna trovare la vista più bassa possibile in grado di materializzare entrambe. A fine iterazione avremo la lista delle viste candidate.
\subsection{Scenari temporali e viste materializzate}
In questi esercizi ancora una volta si chiede di effettuare progettazione logica, ma in aggiunta si chiede di materializzare delle viste e di gestire casi in cui si vogliono prendere in considerazione diversi scenari temporali. Se gli scenari temporali interessanti in una relazione sono solo di un tipo tra 1 e 2, cioè verità storica o attualizzazione, lo schema non subisce alcun tipo di impatto. Nel caso in cui si agisce su più di uno scenario temporale o sulla retrodatazione di entra nel tipo 3, e in questo caso si agisce sullo schema. Si aggiunge nella DT interessata il campo di data inizio e data fine validità e il “master” ossia il vecchio record di riferimento. Per materializzare una vista è necessario aggiungere delle Fact Table in cui le chiavi sono surrogati degli attributi richiesti. I surrogati non compaiono dal nulla, bisogna aggiungere delle tabelle, o facendo snowflake sullo schema originale o semplicemente replicando le tabelle.
\newpage
