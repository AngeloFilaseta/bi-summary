\section{Ciclo di vita di un Data Warehouse}
\subsection{Fattori di Rischio}
Esistono numerosi fattori di rischio che possono portare al fallimento o ad un risultato non soddisfacente nei progetti di data warehousing. I rischi sono legati:
\begin{itemize}
	\item \textbf{alla gestione del progetto}: se non viene effettuato un efficace project management i tempi rischiano di dilatarsi oltre misura e si rischia di non rispettare le aspettative;
	\item \textbf{alle tecnologie}: bisogna essere in grado fin dall'inizio se i requisiti saranno soddisfatti con le tecnologie scelte;
	\item \textbf{ai dati}: i dati che ho sul database operazionali potrebbero non essere sufficienti per ottenere il contenuto informativo desiderato;
	\item \textbf{alla progettazione}: se viene utilizzato su uno stack tecnologico non valido le prestazioni potrebbero non essere soddisfacenti;
	\item \textbf{all'organizzazione}: i sistemi impattano pesantemente sul modo di lavorare delle persone a livelli alti dell'organigramma.
\end{itemize}
\subsection{Approcci di realizzazione}
Esistono due tipi di approccio per la realizzazione di un Data Warehouse.
\subsubsection{Approccio top-down}
Pensare, pianificare e realizzare il Data warehouse nella sua interezza, cioè a partire dall'analisi fatta sui requisiti dell'intera azienda.
\begin{itemize}
	\item \textbf{PRO}:
	\begin{itemize}
		\item I risultati sono spesso ottimi in quanto lo scope è tutta l'azienda. Il Data Warehouse risulterà consistente ed integrato.
	\end{itemize}
	\item \textbf{CONTRO}:
	\begin{itemize}
		\item Costi notevoli.
		\item Analisi e riconciliazioni si tutte le sorgenti è un lavoro estremamente complesso.
		\item La fase di analisi deve essere portata avanti con tutte le aree aziendali, lavoro estremamente lungo e complesso.
		\item Impossibile riconoscere a priori i problemi di ogni area aziendale, il processo di analisi rischia la paralisi.
		\item Impossibilità di creare prototipi che possano far capire l'utilità del sistema, gli utenti rischiano di perdere fiducia nel progetto.
	\end{itemize}
\end{itemize}
\noindent L'approccio Top Down in questo campo non viene più seguito in quanto porta spesso a fallimenti. Il rischio è troppo elevato rispetto ai benefici ottenuti.
\subsection{Approccio bottom-up}
La costruzione del Data Warehouse avviene in maniera incrementale, un Data Mart alla volta. Si lavora a cicli di iterazione, dove ad ogni ciclo si progetterà e si creerà un singolo Data Mart. Il Data Warehouse sarà l'unione di tutti i Data Mart.
\begin{itemize}
	\item \textbf{PRO}:
	\begin{itemize}
		\item Si ottengono risultati concreti in breve tempo.
		\item Costi meno elevati;
		\item Semplicità di realizzazione del singolo Data Mart. Focalizzazione sul singolo problema in esame;
		\item La dirigenza riconosce in fretta l'utilità del progetto, avendo a disposizione risultati in breve tempo;
		\item L'attenzione sul progetto è costante.
	\end{itemize}
	\item \textbf{CONTRO}:
	\begin{itemize}
		\item Si rischia di perdere di vista i bisogni globali dell'azienda, focalizzando la visione solo su domini di interesse parziali.
	\end{itemize}
	Il contro è possibilmente affrontabile effettuando un'analisi iniziale di tipo top down, in un certo senso applicando un approccio ibrido. Una volta definiti i bisogni collettivi è possibile utilizzare l'approccio bottom-up, ma in questo modo si avrà una maggiore visione d'insieme.
\end{itemize}

\noindent Il primo Data Mart da prototipare deve essere quello più strategico per l'azienda, sarà utile anche in una fase futura, in quanto avrà un ruolo centrale e di riferimento per l'intero Data Warehouse, e si deve appoggiare su basi di dati già disponibili e consistenti.

\subsection{Il ciclo di Sviluppo}

\begin{enumerate}
	\item \textbf{Definizione degli obiettivi e pianificazione}: studi delle competenze, dei gruppi di lavoro, analisi dei rischi e delle aspettative, valutazioni dei costi e del valore aggiunto, scelta dell'approccio, definizione dell'architettura, stima delle dimensioni. (FASE TOP-DOWN)
	\item \textbf{Progettazione dell'infrastruttura}: Si analizzano le possibili soluzioni tecnologiche adottabili, si realizza un progetto di massima. (FASE TOP-DOWN)
	\item \textbf{Progettazione e sviluppo dei Data Mart}: Ogni iterazione comporta la creazione di un nuovo Data Mart e di nuove applicazioni, che vengono a mano a mano integrata nel Data Warehouse. (FASE BOTTOM-UP)
\end{enumerate}
