\section{Esercizi rimanenti}
\subsection{Calcolo dimensioni dello schema logico a partire dallo schema di fatto}
L'esercizio fornisce uno schema di fatto, lo scopo e fare la progettazione logica creando la FT e le DT. In aggiunta è necessario calcolare quanti byte occupa la soluzione proposta. In alcuni esercizi viene chiesto di trovare la soluzione che occupa meno spazio, in altre di trovare più di una soluzione o comunque le principali. Per trovare la soluzione meno costosa spesso basta fare snowflaking il più possibile.\newline

\noindent Una volta identificate le DT il calcolo della dimensione è la sommatoria del peso di tutte le tabelle. Per calcolare il peso di una tabella la formula da utilizzare è $peso_tupla \cdot cardinalità$.\newline

\noindent La parte difficile di questo esercizio è capire come strutturare le DT.

\newpage
